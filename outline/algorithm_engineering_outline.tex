\documentclass[8pt,a4paper,fleqn]{article}
\usepackage{extsizes}
\usepackage[left=25mm,right=25mm,top=30mm,bottom=25mm]{geometry}
\usepackage[T1]{fontenc}
\usepackage[utf8]{inputenc}
\usepackage{amsmath,mathtools}
\usepackage{dsfont}
\usepackage{multicol}
\usepackage{times}

\newcommand\setComplex{\mathds{C}}
\newcommand\setReal{\mathds{R}}

\begin{document}
  \hrule
  \begin{center}
  \huge \textbf{Algorithm Engineering: Gliederung}
  \end{center}
  \bigskip
  {\large \begin{minipage}[c]{0.4\textwidth}\flushleft Markus Pawellek \\ markuspawellek@gmail.com\end{minipage} \hfill \begin{minipage}[c]{0.4\textwidth}\flushright\today\end{minipage}}
  \medskip
  \hrule
  \bigskip

  \begin{multicols}{2}
  \section{C++-Basics} % (fold)
  \label{sec:c_basics}
  \begin{itemize}
    \item use C++ Reference (cppreference.com) and the official standard
    \item C++ is based on different paradigms
    \item use cmake to organize projects
    \item use \texttt{nullptr} instead of \texttt{NULL}
    \item C++ random number generation
    \item prefer \texttt{std::cout} over \texttt{std::printf}
    \item include headers that you are using and no headers you are not using
    \item do not write code you do not use, make everything as simple as possible
    \item automize the processes using tools like clang-tidy, clang-format, cpplint, etc.
    \item code is written for humans not for the computer
    \item use Google C++ style guide
    \item do not use variable-length arrays
    \item measuring time
    \item read arguments from the command line
    \item avoid using \texttt{unsigned}-qualifier
    \item avoid manually managing memory with \texttt{new} and \texttt{delete}
    \item rely on RAII-principle
    \item prefer \texttt{enum class} over \texttt{enum}
    \item use \texttt{constexpr} instead of \texttt{enum}
    \item pass an argument by \texttt{const \&}
    \item for readability one declaration per line
    \item for teamwork everything should be written in English with ASCII-characters
  \end{itemize}
  % section c_basics (end)

  \section{Motivation} % (fold)
  \label{sec:motivation}
  \begin{itemize}
    \item worst-case behavior and discrepancies between theory and practice
    \item cycle of development and algorithm engineering
    \item What is algorithm engineering?
    \item Why Algorithm Engineering?
    \item Experiment with Algorithms
    \item Paths to Glory
    \item Engineering Aspects
    \item Robert Pike Rules
    \item Unix Philosophy
  \end{itemize}
  % section motivation (end)

  \section{Version Control with Git} % (fold)
  \label{sec:version_control_with_git}
  \begin{itemize}
    \item Distributed Version Control with Git
    \item How Git stores Data
    \item Interacting with Git
    \item Creating a repository
    \item Adding content
    \item Status
    \item Showing Modifications
    \item Committing
    \item Commit Message
    \item Unstaging
    \item Untracking content
    \item Checkout
    \item Branching
    \item Merging
    \item Branching and Merging
    \item Commit History
    \item Tagging
    \item Interact with Remote Repositories
    \item Managing remote repositories
    \item cloning a repository
    \item fetching from remote repositories
    \item Git reference
  \end{itemize}
  % section version_control_with_git (end)

  \section{Building Software} % (fold)
  \label{sec:building_software}
  \begin{itemize}
    \item configure, build and install
    \item guidelines for building software
    \item make and makefiles
    \item autotools
  \end{itemize}
  % section building_software (end)

  \section{CMake} % (fold)
  \label{sec:cmake}
  \begin{itemize}
    \item Interest over time
    \item cmake
    \item using cmake
    \item example
    \item \texttt{cmake --build .}
    \item \texttt{cmake\_minimum\_required}
    \item \texttt{project}
    \item \texttt{add\_subdirectory}
    \item \texttt{find\_package}
  \end{itemize}
  % section cmake (end)

  \section{High-Performance Motivation} % (fold)
  \label{sec:high_performance_motivation}
  \begin{itemize}
    \item realistic environment
    \item Moore's Law
    \item more Cores
    \item wider vector units
    \item consequences for developers
    \item outlook and history
    \item xeon phi architecture
    \item chip layout
    \item core structure
    \item hardware threads
    \item cache organization
    \item prefetching
    \item vector processing units
    \item performance metrics
    \item reality of parallelization
    \item example
    \item amdahl's law
    \item conclusion
  \end{itemize}
  % section high_performance_motivation (end)

  \section{sofware testing} % (fold)
  \label{sec:sofware_testing}
  \begin{itemize}
    \item what is software testing
    \item why test software
    \item testing levels: unit tests, integration tests
  \end{itemize}
  \subsection{testing tools} % (fold)
  \label{sub:testing_tools}
  \begin{itemize}
    \item unit testing with cmake
    \item continuous integration (CI) using gitlab or github
    \item catch - test driven development in c++: key-features, examples, basics, catch-sections, bdd-style test cases
  \end{itemize}
  % subsection testing_tools (end)
  \subsection{what makes good unit tests} % (fold)
  \label{sub:what_makes_good_unit_tests}
  \begin{itemize}
    \item Step 0 - write tests
    \item properties
    \item correctness
    \item readability
    \item completeness
    \item demonstrability
    \item resiliance: ordering, nonhermeticity, deep dependence
    \item recap: what is the goal?
  \end{itemize}
  % subsection what_makes_good_unit_tests (end)
  % section sofware_testing (end)

  \section{Caches} % (fold)
  \label{sec:caches}
  \begin{itemize}
    \item cache structure
    \item cache attached to core
    \item cache lines
    \item cache miss
    \item intel xeon phi cache
    \item example
    \item writing to cache
    \item example
    \item communication
    \item cache coherency
    \item false sharing
    \item example
    \item scaling
    \item how to avoid false sharing
    \item summary
  \end{itemize}
  % section caches (end)

  \section{Parallelism} % (fold)
  \label{sec:parallelism}
  \begin{itemize}
    \item multiple levels of parallelism
    \item vectorization
    \item current vectorization hardware
    \item vector registers
    \item history
    \item why vectorization
    \item compile will do?
    \item how do we vectorize?
    \item intrinsics review and do it yourself
    \item steps to vectorization
    \item vector-loop requirements
    \item how to think of auto-vectorization
    \item how to instruct the compiler
    \item challenge: loop dependencies: read after write, write after read
    \item aliasing
    \item aliasing and compilers
    \item resolving dependencies
    \item example
    \item caches and alignment
    \item alignment for vectorization
    \item alignment on 512-bit wide registers
    \item how to align: tell the compiler about it
    \item what happens without alignment
    \item alignment - multi-dimensional data structures
    \item padding
    \item strided access
    \item streaming stores
    \item example
    \item structure-of-arrays versus array-of-structures
    \item cache-blocking
    \item blocking-basics, blocking-principle, loop-splitting, loop-interchange
    \item blocking versus non-blocking
    \item example: matrix transposition
    \item example: matrix multiplication
    \item vectorization guidelines
  \end{itemize}
  % section parallelism (end)

  \section{multiple levels of parallelism} % (fold)
  \label{sec:multiple_levels_of_parallelism}
  \begin{itemize}
    \item dot product sequential
    \item dot product vectorized
    \item dot product \texttt{std::thread}
    \item dot product OpenMP
  \end{itemize}
  \subsection{OpenMP} % (fold)
  \label{sub:openmp}
  \begin{itemize}
    \item history
    \item OpenMP
    \item OpenMP-Process
    \item basics
    \item library
    \item example: hello world
    \item execution model
    \item communication
    \item parallel directive
    \item how many threads
    \item work-sharing, for-directive
    \item example: for-directive
    \item sections-directive, example
    \item single-directive, example
    \item synchronization, example
    \item critical-directive, example
    \item what does critical do internally
    \item atomic-directive, example
    \item tasking: concept, directive, clauses, example
    \item when do tasks get executed
    \item taskyield directive
    \item openmp environment
  \end{itemize}
  % subsection openmp (end)
  % section multiple_levels_of_parallelism (end)
  \end{multicols}
\end{document}